\section{Our Approach}

\subsection{Core Pig Latin Features}
\begin{frame}{Core Pig Features}
\begin{itemize}
	\item \textbf{Programs specify queries over relations:} Designed to
	concisely facilitate common data transformation tasks.
	\item \textbf{Programs manipulate aggregate non-atomic types:} e.g. bags
	and tuples.
	\item \textbf{Programs use UDFs from environment.}
	\item \textbf{Programs specify data flow via statements:} A sequence of
	statements define dependencies between queries via var assignments and uses.
	\item \textbf{Programs are parallelizable/distributable:} Part of a long
	history in data flow and query-oriented programming.
\end{itemize}
\end{frame}

\begin{frame}{Conventions}
\centering
	\begin{flushleft}
		T : a type\newline
		S : a type that satisfies schema type\newline
		c : an integer to denote column offset\newline
		x, y : identifiers\newline
		\textGamma \: : Context\newline
 		s : a statement\newline
 		q : a term of query form\newline 
	\end{flushleft}
\end{frame}

\subsection{Logical Plan}
\begin{frame}{Formalism for Logical Plan}
\begin{itemize}
	\item We have hope to create formalisms at each level which reflect all five
	core features.
	\item We present our first attempt of a logical plan grammar and static
	semantics
	\item We believe this reflects 4/5 core features.
\end{itemize}
\end{frame}

\begin{frame}{Grammar: Schema Types}
\centering
	\begin{flushleft}
	$ \texttt{schema} := \hfill \texttt{Schema Types}\hfill$\\
	$ \quad \mid \texttt{STNil}\hfill \texttt{{Unit Type}}\hfill$\\
   	$ \quad \mid \texttt{STConsInt} \hfill \texttt{Function Type}\hfill$\\
    $ \quad \mid \texttt{STConsBag} \hfill \texttt{Predicate Type}\hfill$\\
	\end{flushleft}
\end{frame}

\begin{frame}{Grammar: UDF Types}
\centering
	\begin{flushleft}
	$ \texttt{udf} := \hfill \texttt{User Defined Types}\hfill$\\
	$ \quad \mid \texttt{UDFTFn}\hfill \texttt{UDF Function Type}\hfill$\\
   	$ \quad \mid \texttt{UDFTPred} \hfill \texttt{UDF Predicate Type}\hfill$\\
	\end{flushleft}
\end{frame}

\begin{frame}{Grammar: Loadable Types}
\centering
	\begin{flushleft}
	$ \texttt{loadable\_ty} \::= \hfill \texttt{Loadable Types}\hfill$\\
	$ \quad \mid \texttt{LTSchema}\hfill \texttt{Loadable Schema}\hfill$\\
   	$ \quad \mid \texttt{LTUDF} \hfill \texttt{Loadable UDF}\hfill$\\
	\end{flushleft}
\end{frame}

\begin{frame}{Grammar: Terms}
\centering
	\begin{flushleft}
	$ \texttt{tm} := \hfill \texttt{Terms}\hfill$\\
	$ \quad \mid \texttt{t\_filter}\hfill \texttt{Query Filter}\hfill$\\
   	$ \quad \mid \texttt{t\_foreach} \hfill \texttt{Query ForEach}\hfill$\\
    $ \quad \mid \texttt{t\_group} \hfill \texttt{Query Group}\hfill$\\
    $ \quad \mid \texttt{t\_join} \hfill \texttt{Query Join}\hfill$\\
    $ \quad \mid \texttt{t\_load} \hfill \texttt{Load Statement}\hfill$\\
   	$ \quad \mid \texttt{t\_assign} \hfill \texttt{Assignment Statement}\hfill$\\
    $ \quad \mid \texttt{t\_seq} \hfill \texttt{Sequence of Statements}\hfill$\\
    $ \quad \mid \texttt{t\_store} \hfill \texttt{Store Statement}\hfill$\\
	\end{flushleft}
\end{frame}

\begin{frame}{Grammar: Types}
\centering
	\begin{flushleft}
	$ ty \::= \hfill \textnormal{\emph{Types}}\hfill$\\
	$ \quad \mid TUnit\hfill \textnormal{\emph{Unit Type}}\hfill$\\
   	$ \quad \mid TFn \hfill \textnormal{\emph{Function Type}}\hfill$\\
    $ \quad \mid TPred \hfill \textnormal{\emph{Predicate Type}}\hfill$\\
    $ \quad \mid TNil \hfill \textnormal{\emph{Schema Tuple Terminator}}\hfill$\\
    $ \quad \mid TCons \hfill \textnormal{\emph{Schema Tuple Extension}}\hfill$\\
   	$ \quad \mid TInt \hfill \textnormal{\emph{Atomic Schema Attribute}}\hfill$\\
    $ \quad \mid TBag \hfill \textnormal{\emph{Compound Schema Attribute}}\hfill$\\
	\end{flushleft}
\end{frame}

\begin{frame}{Typing Rules: Queries}
	\begin{mathpar}
		\inferrule* [Right=\textbf{T\_Filter}]
          		{\Gamma \vdash x = S \\ schema\_ty\: S \\ \Gamma \vdash y = TPred \:S} {\Gamma \vdash t\_filter \:x, y \in S }
		\hva \and
		\inferrule* [Right={\textbf{T\_ForEach}}]
          		{schema\_ty\: S1 \\ schema\_ty\: S2 \\\\ 
		\Gamma \vdash x = S1 \\ \Gamma \vdash y = TFn\: S1\:S2} {\Gamma \vdash t\_filter \:x, y \in S }
	\end{mathpar}
\end{frame}

\begin{frame}{Typing Rules: ...Queries}
	\begin{mathpar}
		\inferrule* [Right=\emph{\textbf{T\_Group}}]
          		{\Gamma \vdash x = S1 \\\\ schema\_ty\: S1 \\ schema\_ty\: S2 \\\\ schema\_column S1 c = true \\\\ S2 = TCons TInt (TBag S1)} {\Gamma \vdash t\_group \:x  c \in S2 }
		\hva \and
		\inferrule* [Right=\emph{\textbf{T\_Join}}]
          		{schema\_ty\: S1 \\ schema\_ty\: S2 \\\\ 
		\Gamma \vdash x = S1 \\ \Gamma \vdash y =  S2 \\\\ schema\_column S1 cx = true \\schema\_column S2 cy = true  } {\Gamma \vdash t\_join \:x cx, y cy \in S3 }
	\end{mathpar}
\end{frame}

\begin{frame}{Typing Rules: Statements}
\centering
	\begin{mathpar}
		\inferrule* [Right=\emph{\textbf{T\_Load}}]
          		{\Gamma \vdash x = None \\ loadable\_ty\: T} {\Gamma \vdash t\_load \:x T \in TUnit }
		\hva \and
		\inferrule* [Right=\emph{\textbf{T\_Assign}}]
          		{\Gamma \vdash x = None \\ \Gamma \vdash q = S \\schema_ty S} {\Gamma \vdash t\_assign \:x q \in TUnit }
		\hva \and
		\inferrule* [Right=\emph{\textbf{T\_Store}}]
          		{schema\_ty\: S1 \\ schema\_ty\: S2 \\\\ 
		\Gamma \vdash x = S \\ schema\_ty S} {\Gamma \vdash t\_store \:x \in TUnit }
	\end{mathpar}
\end{frame}

\begin{frame}{Typing Rules: ...Statements}
	\begin{mathpar}
		\inferrule* [Right=\emph{\textbf{T\_SeqLoad}}]
          		{s1 = t\_load x T \\Gamma \vdash s1 \in TUnit \\ \Gamma, x:T \vdash s2 \in TUnit } {\Gamma \vdash t\_seq \:s1 s2 \in TUnit }
		\hva \and
		\inferrule* [Right=\emph{\textbf{T\_SeqAssign}}]
          		{s1 = t\_store x \\ schema\_ty S \\\\\Gamma \vdash x = S \\ \Gamma \vdash s1 \in TUnit \\ \Gamma \vdash s2 \in TUnit} {\Gamma \vdash t\_seq \:s1 s2 \in TUnit }
		\hva \and
		\inferrule* [Right=\emph{\textbf{T\_SeqStore}}]
          		{s1 = t\_store x \\ \Gamma \vdash x = S \\ schema\_ty\: S  \\\\ 
		\Gamma \vdash s1 \in TUnit \\ \Gamma \vdash s2 \in TUnit } {\Gamma \vdash t\_seq \:s1 s2 \in TUnit }
	\end{mathpar}
\end{frame}
